\documentclass[man]{apa6}
\usepackage{lmodern}
\usepackage{amssymb,amsmath}
\usepackage{ifxetex,ifluatex}
\usepackage{fixltx2e} % provides \textsubscript
\ifnum 0\ifxetex 1\fi\ifluatex 1\fi=0 % if pdftex
  \usepackage[T1]{fontenc}
  \usepackage[utf8]{inputenc}
\else % if luatex or xelatex
  \ifxetex
    \usepackage{mathspec}
  \else
    \usepackage{fontspec}
  \fi
  \defaultfontfeatures{Ligatures=TeX,Scale=MatchLowercase}
\fi
% use upquote if available, for straight quotes in verbatim environments
\IfFileExists{upquote.sty}{\usepackage{upquote}}{}
% use microtype if available
\IfFileExists{microtype.sty}{%
\usepackage{microtype}
\UseMicrotypeSet[protrusion]{basicmath} % disable protrusion for tt fonts
}{}
\usepackage{hyperref}
\hypersetup{unicode=true,
            pdftitle={Introduction Draft},
            pdfauthor={Karen Santamaria, Yifan Ma, Caroline Li, \& Jane Bang},
            pdfborder={0 0 0},
            breaklinks=true}
\urlstyle{same}  % don't use monospace font for urls
\usepackage{graphicx,grffile}
\makeatletter
\def\maxwidth{\ifdim\Gin@nat@width>\linewidth\linewidth\else\Gin@nat@width\fi}
\def\maxheight{\ifdim\Gin@nat@height>\textheight\textheight\else\Gin@nat@height\fi}
\makeatother
% Scale images if necessary, so that they will not overflow the page
% margins by default, and it is still possible to overwrite the defaults
% using explicit options in \includegraphics[width, height, ...]{}
\setkeys{Gin}{width=\maxwidth,height=\maxheight,keepaspectratio}
\IfFileExists{parskip.sty}{%
\usepackage{parskip}
}{% else
\setlength{\parindent}{0pt}
\setlength{\parskip}{6pt plus 2pt minus 1pt}
}
\setlength{\emergencystretch}{3em}  % prevent overfull lines
\providecommand{\tightlist}{%
  \setlength{\itemsep}{0pt}\setlength{\parskip}{0pt}}
\setcounter{secnumdepth}{0}
% Redefines (sub)paragraphs to behave more like sections
\ifx\paragraph\undefined\else
\let\oldparagraph\paragraph
\renewcommand{\paragraph}[1]{\oldparagraph{#1}\mbox{}}
\fi
\ifx\subparagraph\undefined\else
\let\oldsubparagraph\subparagraph
\renewcommand{\subparagraph}[1]{\oldsubparagraph{#1}\mbox{}}
\fi

%%% Use protect on footnotes to avoid problems with footnotes in titles
\let\rmarkdownfootnote\footnote%
\def\footnote{\protect\rmarkdownfootnote}


  \title{Introduction Draft}
    \author{Karen Santamaria\textsuperscript{1}, Yifan Ma\textsuperscript{1}, Caroline Li\textsuperscript{1}, \& Jane Bang\textsuperscript{1}}
    \date{}
  
\shorttitle{SDS/PSY 365}
\affiliation{
\vspace{0.5cm}
\textsuperscript{1} Smith College}
\usepackage{csquotes}
\usepackage{upgreek}
\captionsetup{font=singlespacing,justification=justified}

\usepackage{longtable}
\usepackage{lscape}
\usepackage{multirow}
\usepackage{tabularx}
\usepackage[flushleft]{threeparttable}
\usepackage{threeparttablex}

\newenvironment{lltable}{\begin{landscape}\begin{center}\begin{ThreePartTable}}{\end{ThreePartTable}\end{center}\end{landscape}}

\makeatletter
\newcommand\LastLTentrywidth{1em}
\newlength\longtablewidth
\setlength{\longtablewidth}{1in}
\newcommand{\getlongtablewidth}{\begingroup \ifcsname LT@\roman{LT@tables}\endcsname \global\longtablewidth=0pt \renewcommand{\LT@entry}[2]{\global\advance\longtablewidth by ##2\relax\gdef\LastLTentrywidth{##2}}\@nameuse{LT@\roman{LT@tables}} \fi \endgroup}


\DeclareDelayedFloatFlavor{ThreePartTable}{table}
\DeclareDelayedFloatFlavor{lltable}{table}
\DeclareDelayedFloatFlavor*{longtable}{table}
\makeatletter
\renewcommand{\efloat@iwrite}[1]{\immediate\expandafter\protected@write\csname efloat@post#1\endcsname{}}
\makeatother

\authornote{

Correspondence concerning this article should be addressed to Karen Santamaria, 1 Chapin Way, Unit 7766, Northampton, MA 01063. E-mail: \href{mailto:sanmakaren@gmail.com}{\nolinkurl{sanmakaren@gmail.com}}}



\begin{document}
\maketitle

Researchers conceptualized families as complex dynamic systems that were composed of various interactive components and were organized on levels such as the individual level, the dyadic level, and the nuclear family system. Within a family, the dyadic relationship developed in a recursive or iterative fashion and influenced the adjustment of each member of the family and the family as a whole (Van Geert \& Lichtwarck-Aschoff, 2005). Therefore, we may expect that dysregulation or deterioration in one relationship may be negatively associated with other relationships and the process may be conditional dependent in time. While previous studies have shown that conflicts in parental relationships predict a lower quality of parent-child relationships (Cummings \& Davies, 1994; Klausli \& Tresch Owen, 2011; Laurent, Kim, \& Capaldi, 2008), the majority of the existing literature dedicated to the association between biological parents' relationship and the parent-child relationship in heterosexual families. Consequently, little is known about how the relationship within lesbian couple and gay couple (LG) dyads may shape these parents' attachment to their children. Furthermore, while relationship-quality is assessed, little is known about how anxiety and depression play a role in moderating or mediating the relationship between relationship-quality and attachment.

Children's attachment is one of the crucial factors in predicting future misconduct and maladjustment. One study found that infant attachment security moderated the association between parenting in preschool and later aggressive behavior among children at high risk for developing conduct problems (Cyr, Pasalich, McMahon, \& Spieker, 2014). Furthermore, other studies have found that behavioral issues during childhood were a significant predictor for behavioral issues during adolescence and adulthood (Fergusson, John Horwood, \& Ridder, 2005; Reef, Diamantopoulou, Meurs, Verhulst, \& Ende, 2011). Thus, examining factors related to attachment is important because insecure attachment may have lifelong negative effects.

In adoptive families, children may already come with a past where they may have been exposed to adverse emotional and physical experiences in adoptive families (Van IJzendoorn \& Juffer, 2006). Meanwhile, the transition into adoptive parenthood is a stressful time marked by many changes in the daily lives of these parents. Some studies supported increased psychological vulnerability related to adoption (Brodzinsky, Schechter, \& Brodzinsky, 2014). However, research has also shown that there was no difference in the quality of mother and infant attachment between adoptive families and biological families (Singer, Brodzinsky, Ramsay, Steir, \& Waters, 1985). Additionally, other literature has indicated that adoptive families had more positive expectations and experiences than biological parents (Levy-Shiff, Goldshmidt, \& Har-Even, 1991). The mix findings highlighted the importance of examining the parent-child attachment in adoptive families. Moreover, Goldberg and Smith (2009) explored the change in perceived parenting skills among lesbian, gay, and heterosexual couples and found that gay men perceived themselves as having the largest increase in skill while lesbians perceived themselves as having the least. This study indicated that the experiences of LG parents differ from the experiences of heterosexual parents. As a result, the differences in the attachment in heterosexual, lesbian, and gay parents are of interest to investigate.

Studies of intimate relationship and psychopathology have established a bidirectional association of relationship functioning and individual mental health. While relationship problems may act as interpersonal stressors that increase the likelihood of a person developing mental disorders, mental health issues may also be accompanied by changes in relationships that are difficult for the partner to accommodate (Whisman \& Baucom, 2012). Following this principle, one may speculate that anxiety and depressive symptoms may play roles in the association between love between parents and parent-child attachment. Since studies have shown the essential role of a close, warm, and supportive parent-child relationships for children's healthy development (Gao \& Cummings, 2019), it is critical to explore the relationship between love (between parents) and adoptive parents' attachment to adoptees across time and the roles of anxiety and depressive symptoms.

Previous studies have established the impact of marital quality on various outcomes. For instance, a study found out that marital quality had significant influences on health trajectories in the general population; negative marriage experiences accelerated declines in self-rated health over time, and positive marriage experiences decreased over time. Moreover, the effects of marital quality on health are similar between men and women (Umberson, Williams, Powers, Liu, \& Needham, 2006). In addition, the association between marital quality and the parent-child relationship has been widely studied by scholars. Research indicated, for example, that fathers who were maritally less satisfied acted negatively to their daughters regardless of daughters' behaviors. However, there was no evidence for a compensatory bond between less maritally satisfied parents and their same-sex children (Kerig, Cowan, \& Cowan, 1993). Dickstein, Seifer, and Albus (2009) revealed the link between the quality of couples interaction, family functioning, and infant-mother attachment. They found that the quality of couples interaction predicted family functioning which further predicted infant-mother attachment. When individuals reported a higher level of emotional quality with their spouse, they also reported a better relationship with their child (Gao \& Cummings, 2019). Another study supported that that marriage happiness predicted parent-child relationship problems, divorce, and parental affection toward children (Amato \& Booth, 1996). Given the above-mentioned findings, we hypothesized that parental relationship is positively associated with parent-child attachment and will increase the level of attachment across time. Since there is a lack of studies that examine the actor-partner effect in the association, we sought to explore how people's relationship with their spouse is associated with the partner's attachment with their children. Since relationships within a family are interconnected, we expected that love (between parents) would be positively associated with the partner's attachment to children.

Scholars have examined the association between mental health factors such as anxiety and depressive symptoms and dyadic relationships or parent-child attachment. For example, poor dyadic adjustment is found to be associated with both anxiety and depressive symptoms (Stevenson-Hinde, Shouldice, \& Chicot, 2011). Stevenson-Hinde, Curley, Chicot, and Jóhannsson (2007) examined anxiety within families. They revealed that maternal anxiety was significantly associated with mothers' and fathers' independent ratings of marital satisfaction and to fathers' own anxiety and depression while fathers' anxiety was significantly related to mothers' ratings of marital satisfaction and to maternal anxiety. Meanwhile, Stevenson-Hinde et al. (2011) found that both maternal anxiety and children's behavior inhibition were negatively related to the ratings of children's attachment and maternal anxiety was a significant predictor for insecure attachment. A study that compared the mother-child attachment style in depressed and non-depressed mothers reported a higher incidence of insecure attachment model in the depressed-mother group compared to the healthy controls (Santona et al., 2015). Despite these established links, none of the studies we found have investigated the interplay of these three factors. According to the previous findings, we hypothesized that the relationship between love and parent-child attachment would be mediated by anxiety and depressive symptoms. We also decided to incorporate the longitudinal aspect to our model because Goldberg, Moyer, and Kinkler (2013) have shown that a parent's perceived attachment with their children varies over time, and both parent-related factors and child-related factors could influence it. As a result, we would also like to predict that the relationship quality between parents at earlier phases in the study will predict parent-child attachment in later phases.

Previous research examining gay-father families found that parenting stress could predict child externalizing problems and gay fathers have a lower level of parenting stress compared with heterosexual parents (Golombok et al., 2014). We hypothesized that different family structures will result in different correlations between love and parent-child attachment.

Given what previous research has suggested, such as; there are long term effects of attachment security for a child (Cyr et al., 2014; Fergusson et al., 2005; Reef et al., 2011); that dyadic relationships have a ripple or recursive effect on other relationships (Van Geert \& Lichtwarck-Aschoff, 2005); the effect of marital quality may influence health (Umberson et al., 2006); and the associations between anxiety and depressive symptoms and dyadic relationships or parent-child attachment (Stevenson-Hinde et al., 2011), we found it important to analyze how love experienced and expressed corresponds with parent-child attachment. In addition, since only limited research has focused on the influence of different family structures on the correlation between love and perceived attachment, we decided to incorporate the factor of family structure into our model. Our first hypothesis was that love correlates with attachment and this correlation changes with respect to time point. We also assumed that the correlation between love and attachment is different across family type. Furthermore, we predicted the association between love and attachment is bigger for LG couples than it is for heterosexual couples. Within all these hypotheses, we were also interested in finding if anxiety and/or depression plays a role in moderating or mediating these relationships.

\newpage

\hypertarget{references}{%
\section*{References}\label{references}}
\addcontentsline{toc}{section}{References}

\hypertarget{refs}{}
\leavevmode\hypertarget{ref-amato1996prospective}{}%
Amato, P. R., \& Booth, A. (1996). A prospective study of divorce and parent-child relationships. \emph{Journal of Marriage and the Family}, 356--365.

\leavevmode\hypertarget{ref-brodzinsky2014children}{}%
Brodzinsky, D. M., Schechter, D., \& Brodzinsky, A. B. (2014). Children's knowledge of adoption: Developmental changes and implications for adjustment. In \emph{Thinking about the family} (pp. 225--252). Psychology Press.

\leavevmode\hypertarget{ref-cummings1994children}{}%
Cummings, E. M., \& Davies, P. (1994). \emph{Children and marital conflict: The impact of family dispute and resolution.} Guilford Press.

\leavevmode\hypertarget{ref-study3}{}%
Cyr, M., Pasalich, D. S., McMahon, R. J., \& Spieker, S. J. (2014). The longitudinal link between parenting and child aggression: The moderating effect of attachment security. \emph{Child Psychiatry \& Human Development}, \emph{45}(5), 555--564.

\leavevmode\hypertarget{ref-dickstein2009maternal}{}%
Dickstein, S., Seifer, R., \& Albus, K. E. (2009). Maternal adult attachment representations across relationship domains and infant outcomes: The importance of family and couple functioning. \emph{Attachment \& Human Development}, \emph{11}(1), 5--27.

\leavevmode\hypertarget{ref-fergusson2005show}{}%
Fergusson, D. M., John Horwood, L., \& Ridder, E. M. (2005). Show me the child at seven: The consequences of conduct problems in childhood for psychosocial functioning in adulthood. \emph{Journal of Child Psychology and Psychiatry}, \emph{46}(8), 837--849.

\leavevmode\hypertarget{ref-gao2019understanding}{}%
Gao, M. M., \& Cummings, E. M. (2019). Understanding parent--child relationship as a developmental process: Fluctuations across days and changes over years. \emph{Developmental Psychology}.

\leavevmode\hypertarget{ref-goldberg2013lesbian}{}%
Goldberg, A. E., Moyer, A. M., \& Kinkler, L. A. (2013). Lesbian, gay, and heterosexual adoptive parents' perceptions of parental bonding during early parenthood. \emph{Couple and Family Psychology: Research and Practice}, \emph{2}(2), 146.

\leavevmode\hypertarget{ref-goldberg2009perceived}{}%
Goldberg, A. E., \& Smith, J. Z. (2009). Perceived parenting skill across the transition to adoptive parenthood among lesbian, gay, and heterosexual couples. \emph{Journal of Family Psychology}, \emph{23}(6), 861.

\leavevmode\hypertarget{ref-golombok2014adoptive}{}%
Golombok, S., Mellish, L., Jennings, S., Casey, P., Tasker, F., \& Lamb, M. E. (2014). Adoptive gay father families: Parent--child relationships and children's psychological adjustment. \emph{Child Development}, \emph{85}(2), 456--468.

\leavevmode\hypertarget{ref-kerig1993marital}{}%
Kerig, P. K., Cowan, P. A., \& Cowan, C. P. (1993). Marital quality and gender differences in parent-child interaction. \emph{Developmental Psychology}, \emph{29}(6), 931.

\leavevmode\hypertarget{ref-klausli2011exploring}{}%
Klausli, J. F., \& Tresch Owen, M. (2011). Exploring actor and partner effects in associations between marriage and parenting for mothers and fathers. \emph{Parenting}, \emph{11}(4), 264--279.

\leavevmode\hypertarget{ref-laurent2008prospective}{}%
Laurent, H. K., Kim, H. K., \& Capaldi, D. M. (2008). Prospective effects of interparental conflict on child attachment security and the moderating role of parents' romantic attachment. \emph{Journal of Family Psychology}, \emph{22}(3), 377.

\leavevmode\hypertarget{ref-levy1991transition}{}%
Levy-Shiff, R., Goldshmidt, I., \& Har-Even, D. (1991). Transition to parenthood in adoptive families. \emph{Developmental Psychology}, \emph{27}(1), 131.

\leavevmode\hypertarget{ref-reef2011developmental}{}%
Reef, J., Diamantopoulou, S., Meurs, I. van, Verhulst, F. C., \& Ende, J. van der. (2011). Developmental trajectories of child to adolescent externalizing behavior and adult dsm-iv disorder: Results of a 24-year longitudinal study. \emph{Social Psychiatry and Psychiatric Epidemiology}, \emph{46}(12), 1233--1241.

\leavevmode\hypertarget{ref-santona2015maternal}{}%
Santona, A., Tagini, A., Sarracino, D., De Carli, P., Pace, C. S., Parolin, L., \& Terrone, G. (2015). Maternal depression and attachment: The evaluation of mother--child interactions during feeding practice. \emph{Frontiers in Psychology}, \emph{6}, 1235.

\leavevmode\hypertarget{ref-singer1985mother}{}%
Singer, L. M., Brodzinsky, D. M., Ramsay, D., Steir, M., \& Waters, E. (1985). Mother-infant attachment in adoptive families. \emph{Child Development}, 1543--1551.

\leavevmode\hypertarget{ref-stevenson2007anxiety}{}%
Stevenson-Hinde, J., Curley, J. P., Chicot, R., \& Jóhannsson, C. (2007). Anxiety within families: Interrelations, consistency, and change. \emph{Family Process}, \emph{46}(4), 543--556.

\leavevmode\hypertarget{ref-stevenson2011maternal}{}%
Stevenson-Hinde, J., Shouldice, A., \& Chicot, R. (2011). Maternal anxiety, behavioral inhibition, and attachment. \emph{Attachment \& Human Development}, \emph{13}(3), 199--215.

\leavevmode\hypertarget{ref-umberson2006you}{}%
Umberson, D., Williams, K., Powers, D. A., Liu, H., \& Needham, B. (2006). You make me sick: Marital quality and health over the life course. \emph{Journal of Health and Social Behavior}, \emph{47}(1), 1--16.

\leavevmode\hypertarget{ref-van2005dynamic}{}%
Van Geert, P. L., \& Lichtwarck-Aschoff, A. (2005). A dynamic systems approach to family assessment. \emph{European Journal of Psychological Assessment}, \emph{21}(4), 240--248.

\leavevmode\hypertarget{ref-van2006emanuel}{}%
Van IJzendoorn, M. H., \& Juffer, F. (2006). The emanuel miller memorial lecture 2006: Adoption as intervention. Meta-analytic evidence for massive catch-up and plasticity in physical, socio-emotional, and cognitive development. \emph{Journal of Child Psychology and Psychiatry}, \emph{47}(12), 1228--1245.

\leavevmode\hypertarget{ref-whisman2012intimate}{}%
Whisman, M. A., \& Baucom, D. H. (2012). Intimate relationships and psychopathology. \emph{Clinical Child and Family Psychology Review}, \emph{15}(1), 4--13.


\end{document}
